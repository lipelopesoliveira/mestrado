%% abtex2-modelo-include-comandos.tex, v-1.9.6 laurocesar
%% Copyright 2012-2016 by abnTeX2 group at http://www.abntex.net.br/ 
%%
%% This work may be distributed and/or modified under the
%% conditions of the LaTeX Project Public License, either version 1.3
%% of this license or (at your option) any later version.
%% The latest version of this license is in
%%   http://www.latex-project.org/lppl.txt
%% and version 1.3 or later is part of all distributions of LaTeX
%% version 2005/12/01 or later.
%%
%% This work has the LPPL maintenance status `maintained'.
%% 
%% The Current Maintainer of this work is the abnTeX2 team, led
%% by Lauro César Araujo. Further information are available on 
%% http://www.abntex.net.br/
%%
%% This work consists of the files abntex2-modelo-include-comandos.tex
%% and abntex2-modelo-img-marca.pdf
%%

% ---
% Este capítulo, utilizado por diferentes exemplos do abnTeX2, ilustra o uso de
% comandos do abnTeX2 e de LaTeX.
% ---

\chapter{Conclusão}
\label{conclusao}
	
	Com base nos resultados apresentados ao longo desta dissertação, é possível retomar as perguntas levantadas na seção de objetivos e apresentar uma resposta.
	
	\begin{enumerate}
		


		\item[$1^{\circ}$:] \textbf{É possível utilizar métodos DFT para predizer a estrutura e propriedades de diferentes alótropos de carbono?}
			
			Sim. Os resultados apresentados no \autoref{calculos_exp} permitem afirmar que os cálculos de estrutura eletrônica em nível DFT apresentam um bom resultado na predição da estrutura e propriedades, como \textit{band gap} e módulo Bulk por exemplo, dos alótropos de carbono que se têm dados experimentais para serem comparados. Os funcionais de troca-correlação avaliados apresentaram diferentes desempenhos, tendo o funcional PBE sido o que apresentou os resultados mais consistentes com os dados experimentais. Além disso, para sistemas cujas interações intermoleculares possuem um papel estrutural chave, como no grafite, uma correção de energia dispersiva deve ser adicionada aos funcionais GGA para garantir predições acuradas. A correção do tipo Grimme-D3 apresentou um melhor resultado, sendo portanto a indicada para cálculos futuros.
	
		\item[$2^{\circ}$:] \textbf{É possível propor novas estruturas alotrópicas metaestáveis para o carbono baseadas em motivos estruturais moleculares}
			
			Sim. No \autoref{chap:novos_alotropos} foram apresentadas duas novas formas alotrópicas para o carbono, denominadas Spiro-Carbon e ABF-Carbon, contendo o motivo estrutural derivado da molécula spiropentadieno. Os resultados apresentados mostram que ambas as estruturas correspondem a um mínimo na superfície de potencial, são mecanicamente estáveis e apresentam menor energia de formação que outros alótropos já obtidos experimentalmente, como o T-Carbon, ou ainda elusivos como o T-II-Carbon e Y-Carbon.
	
		\item[$3^{\circ}$:] \textbf{É possível utilizar uma forma alotrópica já existente, como por exemplo o carbino, como "\textit{blocos de construção}" para novos alótropos de carbono?}
		
			Sim. No \autoref{alotropos_carbina} foi mostrado que alótropos já conhecidos, como o grafeno e o bct-4, podem ser vistos como cadeias distorcidas do carbino conectadas de diferentes maneiras. Além disso, utilizando essa estratégia é possível construir um espaço amostral de diferentes estruturas possíveis o que pode levar à descoberta de diversas novas formas alotrópicas estáveis para o carbono.
		
	\end{enumerate}
