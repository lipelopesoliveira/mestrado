%% abtex2-modelo-include-comandos.tex, v-1.9.6 laurocesar
%% Copyright 2012-2016 by abnTeX2 group at http://www.abntex.net.br/ 
%%
%% This work may be distributed and/or modified under the
%% conditions of the LaTeX Project Public License, either version 1.3
%% of this license or (at your option) any later version.
%% The latest version of this license is in
%%   http://www.latex-project.org/lppl.txt
%% and version 1.3 or later is part of all distributions of LaTeX
%% version 2005/12/01 or later.
%%
%% This work has the LPPL maintenance status `maintained'.
%% 
%% The Current Maintainer of this work is the abnTeX2 team, led
%% by Lauro César Araujo. Further information are available on 
%% http://www.abntex.net.br/
%%
%% This work consists of the files abntex2-modelo-include-comandos.tex
%% and abntex2-modelo-img-marca.pdf
%%

% ---
% Este capítulo, utilizado por diferentes exemplos do abnTeX2, ilustra o uso de
% comandos do abnTeX2 e de LaTeX.
% ---
\chapter{Revisão Bibliográfica}

\section{Liberação controlada e sustentada de feromônios}

Atualmente, o controle e gerenciamento de insetos e pragas em plantações é extremamente dependente da utilização de agrotóxicos. Mesmo apresentando diversos problemas associados a utilização desses químicos, como por exemplo poluição ambiental, contaminação dos produtos, desenvolvimento de resistência nas pragas contaminação de outros animais dentre outros, esses compostos ainda são utilizados em escala de centenas de toneladas em plantações por todo o mundo \cite{rigotto2014pesticide} \cite{lewis2015pesticide}.

Uma das alternativas mais elegantes aos agrotóxicos é a utilização de feromônios para monitorar a população de pragas, atrapalhar/impedir sua reprodução ou atraí-las para armadilhas. Devido a seu baixo custo, alta sensibilidade/especificidade e seu baixíssimo impacto ambiental, a utilização de feromônios é uma das alternativas mais promissoras tanto em pequenas quanto em grandes plantações \cite{howse2013insect} \cite{witzgall2010sex}. 

Entretanto, o fator mais crítico para sua utilização é o controle da dose liberada por longos períodos de tempo. Pouco têm se pesquisado sobre novas formas de liberação controlada e sustentada de feromônios, e ainda hoje não existe uma metodologia de auto rendimento para resolver esse problema \cite{el2006potential}.

\citeauthoronline{atterholt1998study}, em \citeyear{atterholt1998study}, mostraram que biopolímeros derivados de amido, soro de leite e soja não apresentam taxas de liberação de feromônios lentas e constantes o suficientes para serem aplicados como matriz de liberação de feromônios. 

\citeauthoronline{tomaszewska2005evaluation}, em \citeyear{tomaszewska2005evaluation}, na tentativa de desenvolver sistema simples e confiável para medir a taxa de liberação de feromônios compararam alguns sistemas comerciais disponíveis e observaram taxas de liberação erráticas e variáveis entre esses sistemas.

Em \citeyear{moreno2016single}, \citeauthoronline{moreno2016single} demonstraram a utilização de materiais híbridos lamelares, derivados de MOFs, como matriz para liberação controlada de um feromônio emitido por \textit{Aonidiella aurantii}. Apesar de ter demonstrado uma capacidade de carga de feromônios alta (cerca de 25\% em peso), esses materiais só demonstraram uma capacidade de liberação controlada da primeira semana de testes, e mantiveram uma grande quantidade de feromônio residual \cite{moreno2016single}. 

Em \citeyear{yoon2017sustainable}, \citeauthoronline{yoon2017sustainable} demonstraram a utilização de cera de abelhas para confecção de matrizes para liberação sustentada de feromônios. Apesar de ter conseguido taxas de liberação aproximadamente constantes num período de até 100 dias, devido à alta solubilidade dos feromônios da cera, a quantidade liberada foi muito pequena. Além disso, esse tipo de material apresenta muito pouca possibilidade de modulação e resistência à condições extremas \cite{yoon2017sustainable}.

\citeauthoronline{stipanovic2004microparticle} estudando micropartículas para liberação de feromônios demonstraram que o controle da área superficial e volume de poro são os fatores mais importantes para controlar a taxa de liberação. 

\citeauthoronline{munoz2001zeolites} em um artigo muito completo publicado em \citeyear{munoz2001zeolites} sobre a utilização de zeolitas como matriz para liberação de feromônios mostraram que a polaridade e polarizabilidade da matriz, junto com as interações intermoleculares que podem ocorrer entre moléculas e matriz podem ter uma grande influência na taxa e liberação dessas moléculas.

\citeauthoronline{slodowicz2017physicochemical} publicaram em \citeyear{slodowicz2017physicochemical} um estudo bem completo sobre as propriedades fisico-qúmicas de diversas ceras naturais e suas influências na taxa de liberação de feromônios. Apesar de terem conseguido uma performance relativamente boa de algumas ceras, ainda há muito a ser melhorado. 
