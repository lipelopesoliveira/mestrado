%% abtex2-modelo-include-comandos.tex, v-1.9.6 laurocesar
%% Copyright 2012-2016 by abnTeX2 group at http://www.abntex.net.br/ 
%%
%% This work may be distributed and/or modified under the
%% conditions of the LaTeX Project Public License, either version 1.3
%% of this license or (at your option) any later version.
%% The latest version of this license is in
%%   http://www.latex-project.org/lppl.txt
%% and version 1.3 or later is part of all distributions of LaTeX
%% version 2005/12/01 or later.
%%
%% This work has the LPPL maintenance status `maintained'.
%% 
%% The Current Maintainer of this work is the abnTeX2 team, led
%% by Lauro César Araujo. Further information are available on 
%% http://www.abntex.net.br/
%%
%% This work consists of the files abntex2-modelo-include-comandos.tex
%% and abntex2-modelo-img-marca.pdf
%%

% ---
% Este capítulo, utilizado por diferentes exemplos do abnTeX2, ilustra o uso de
% comandos do abnTeX2 e de LaTeX.
% ---

\chapter{Objetivos}

O principal objetivo deste trabalho é explorar a utilização de métodos de cálculos baseados na Teoria do Funcional da Densidade (DFT) para diferentes estruturas alotrópicas do carbono, prevendo a existência de novas estruturas e suas possíveis propriedades estruturais, eletrônicas e vibracionais. 

Esse trabalho pode ser resumido em três principais perguntas:

\begin{enumerate}
	
	\item[1$^{\circ}$:] É possível utilizar métodos DFT para predizer a estrutura e propriedades de diferentes alótropos de carbono?
		\subitem $\bullet$ Qual o melhor funcional DFT?
		\subitem $\bullet$ Correções de energia de dispersão são necessária?
	
	\item[2$^{\circ}$:] É possível propor novas estruturas alotrópicas metaestáveis para o carbono baseadas em motivos estruturais moleculares?
		\subitem $\bullet$ É possível propor alótropos de carbono com base na estrutura do spiropentadieno?
		\subitem $\bullet$ É possível propor alótropos combinando motivos estruturais, como o motivo spiro, e átomos de carbono com uma hibridização específica?

	\item[3$^{\circ}$:] É possível utilizar formas alotrópicas já existentes, como por exemplo o carbino, como “blocos de construção” para novos alótropos de carbono?
	
\end{enumerate}
